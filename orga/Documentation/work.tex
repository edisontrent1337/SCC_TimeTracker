
% Angaben zum Team, Vorgehensweise

\section{Arbeitsablauf}

\subsection{Team}

Das Entwicklerteam besteht aus den Studenten Sinthujan Thanabalasingam  (Master Informatik, 4. Semester) und Wieland Strauß (Diplom Informatik, 7. Semester). 

Sinthujan beschäftigte sich schwerpunktmäßig mit Kozeptionierung und Implementation der Architektur des Backends sowie Entwurf und dem Design des Frontends. Bei Wieland lag der Fokus bei der Implementation im Pair-Programming, der Dokumentation sowie allen organisatorischen Aufgaben.

Die Implementation erfolgte phasenweise im Pair-Programming. Zu Beginn wurde der Fokus auf die benötigte Backend-Funktionalität gelegt, es gab jedoch vorwiegend keine spezifische Aufteilung zwischen Backend- und Frontend-Entwicklung.  

\subsection{Vorgehensweise}

Nach der Ideenfindung wurde zuerst die Architektur entworfen. 
Die Entwicklung wurde nahezu vollständig nach dem Prinzip des Test Driven Development (TDD) betrieben. 
Eine API-Definition wurde zu Beginn mit \textit{Swagger} erstellt und parallel zur Entwicklung des Backends erweitert, wenn neue Anforderungen identifiziert wurden. Da die vorgeschlagene Architektur sich Micro-Services zu nutze macht, wurde bereits während der frühen Entwicklungs- und Testphase Docker als Containertechnik eingesetzt.
Parallel zur Backend-Entwicklung wurde ein auf React basierendes Frontend geschaffen.

Zuletzt wurde die Anwendung auf einem V-Server von DigitalOcean ausgerollt, mit HTTPS abgesichert und unter der Domain \url{https://iamtrent.de} verfügbar gemacht.

Der Arbeitsablauf organisierte sich aus mehreren Treffen. Bei diesen wurde die Planung durchgeführt, ein Großteil der Anwendung geschrieben und Aufgaben für die selbstständige Durchführung festgelegt. Aufgrund der unterschiedlichen Expertise der Entwickler wurde Pair-Programming praktiziert. Weite Teile der Backend-Entwicklung fanden auch nach Absprache bei den Treffen selbstständig, zum Beispiel zu Hause statt.  


