% Idee, usw. 
\section{Über TimeTracker}

\subsection{Inhalt der Anwendung}
TimeTracker ist eine Anwendung, die es ermöglicht, Zeitinvestitionen zu managen. Ein registrierter Nutzer kann das System nutzen, um den Zeitaufwand bestimmter Aktivitäten aufzuzeichnen. Er kann dazu Activities anlegen und Stoppuhren starten, welche die Dauer der Aufgaben aufzeichnen.

Dem Nutzer wird eine Übersicht über seine aufgenommenen Zeiten angeboten. Dadurch bekommt er einen besseren Überblick darüber, wie viel Zeit er in welche Aufgaben investiert hat. Das Tool soll den Nutzer dabei unterstützen, Hotspots oder Muster in seinem Verhalten zu identifizieren um gegebenenfalls seine Prioritäten anzupassen.

Zusätzlich bietet die Plattform eine anonymisierte globale Statistik, die alle Zeitinvestitionen aller Nutzer repräsentiert und nur von eingeloggten Nutzern eingesehen werden kann.

%Einordnung 
\subsection{Einordnung der Arbeit}
Die Anwendung wurde als Prüfungsvorleistung im Rahmen einer praktischen Übung der Lehrveranstaltung \textit{Service and Cloud Computing} des Lehrstuhls Rechnernetze der TU Dresden entwickelt. Dozentin und Betreuerin ist Frau Dr.-Ing. Iris Braun. 