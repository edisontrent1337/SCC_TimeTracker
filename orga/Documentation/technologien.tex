%  Verwendete Plattform /Software (Installationshinweise, Versionen)
% Schnittstellenbeschreibung des Web Services (WSDL/WADL/Swagger, …​)
% Umsetzung, Technologien, Sicherheitskonzepte, API

\section{Technologien}

\subsection{Übersicht verwendeter Technologien, Software  \newline und Formate}

Die in der Anwendung implementierten Technologien aus Tabelle \ref{tab:Technologien}  sind in Abbildung \ref{fig:TechStack} noch einmal schematisch zugeordnet. 
Die Anwendung (blau) läuft in einem Docker-Container (dunkelblau). Über die API (grün) kommuniziert die Weboberfläche (hellblau) mit der Anwendung. Die sich in Planung befindende App hat den gleichen Zugriffspunkt. 

Eine Zuordnung der Technologien zur Architektur erfolgt in Abbildung \ref{fig:ArchitekturTech}

Anmerkung: Die Andoid-App wurde nicht im Rahmen der Lehrveranstaltung fertig gestellt und ist somit noch ausstehend. Da sie aber für eine spätere Nutzung angedacht ist und sie ebenfalls über die REST-Schnittstelle kommuniziert wurde sie im Entwurf berücksichtigt. 

% Tabelle 
\renewcommand{\arraystretch}{1.3} % Abstände zwischen den Spalten 
\begin{table}[H] 
	% \small
	\centering
	\caption{Technologien}
	\begin{tabularx}{\textwidth}{l|c|X}
		\textbf{Name} & \textbf{Version} & \textbf{Verwendung} \\
		\hline
		Java & JDK 1.8.0 & Verwendete Programmiersprache \\
		Maven & 3.1.0  & Build-Management-Tool \\
		Spring Boot & 2.0.2 & Java Framework für Backends \\
		MySQL & 5.7.0 & Relationale Datenbank \\
		Netflix Zuul & 1.3.1 & Edge Service für dynamisches Routen, \newline Monitoring und Sicherheit  \\
		Netflix Eureka & 1.9.2 &  Service-Registry   \\
		JSON & - & Datenaustauschformat \\
		REST & - & Programmierparadigma für Webservices \\
		React & 16.1.1 & JavaScript Bibliothek für User Interfaces  \\
		single-spa & 2.6.0 & JavaScript Framework für Front-End Microservices \\
		OpenAPI  (Swagger) & 3.0 & Definition der REST-Schnittstelle; Automatisches Generieren von Java-Interfaces  \\
		Postman & 6.5.3 & API Development Environment, \newline Testen der API \\
		Docker & 18.09.0 & Umgebung für Container, Deployment \\
	\end{tabularx}
	\label{tab:Technologien}
\end{table}

\begin{figure}[H]
	\hspace{-1.5cm}
	\includegraphics[width=1.15\linewidth]{technology_stack.png}
	\caption{Technologie Stack}
	\label{fig:TechStack}
\end{figure}

\subsection{Architektur}

%Beschriebender Text?

\begin{figure}[H]
	\hspace{-1.5cm}
	\includegraphics[width=1.15\linewidth]{SCC_Architecture-99final.png}
	\caption{Architektur von TimeTracker}
	\label{fig:Architektur}
\end{figure}



\begin{figure}[H]
	\hspace{-1.5cm}
	\includegraphics[width=1.15\linewidth]{SCC_Archtitecture_Technology.png}
	\caption{Architektur mit zugeordneten Technologien}
	\label{fig:ArchitekturTech}
\end{figure}





\paragraph{Die Funktionen der Services}
\begin{itemize}
	\item User Service
	\begin{itemize}
		\item Registrierung
		\item Login
	\end{itemize}
	\item Timing Service
	\begin{itemize}
		\item Anlegen/ Abruf von Aktivitäten
		\item Anlegen/ Abruf von Records
		\item Abruf von Statistiken
	\end{itemize}
	\item Frontend Service
	\begin{itemize}
		\item Aufruf von UI
	\end{itemize}
	\item Service Registry
	\begin{itemize}
		\item Hält Services vor
	\end{itemize}
	\item API Gateway
	\begin{itemize}
		\item Nutzt Service Registry
		\item Mapping der Requests auf Service
	\end{itemize}
\end{itemize}

Durch die Micro Service Architektur war es möglich rolling Updates zu machen, was insbesondere im Bereich der UI zum Einsatz kam. 

\subsection{Schnittstellen}
% Links von Server mit http://editor.swagger.io/ oder https://swagger.io/tools/swagger-inspector/ or https://swagger.io/tools/swagger-ui/

Die API wurde nach dem OpenAPI-Standard entwickelt und generierte automatisch ein Interface in Java. Eine Auflistung aller Methoden ist über die folgenden Links zu erreichen. Zusätzlich befindet sich im Anhang (Seite \pageref{Anhang}) ein Ausdruck der automatisch generierten Dokumentation.

\begin{itemize}
	\item frontend-service: \newline  \url{https://app.swaggerhub.com/apis/SCC_Group4/frontend-service/}
	\item timing-service: \newline \url{https://app.swaggerhub.com/apis/SCC_Group4/timing-service/}
	\item user-service: \newline \url{https://app.swaggerhub.com/apis/SCC_Group4/user-service/}
\end{itemize}

\subsection{Sicherheit}

Die Sicherheit wurde mittels \textbf{JSON Web Token (JWT)} und umgesetzt. Dabei erhält der Benutzer beim Login einen Token, mittels derer er sich gegenüber den Diensten authentifizieren kann. Das Token wird im localStorage des Browsers gespeichert und bei jedem Request im Header mitgeführt. 

Des Weiteren wird die Sicherheit der Verbindung mittels der Transportverschlüsselung \textbf{HTTPS} gewährleistet. 
