% Feedback + Kritik am Praktikum

\section{Feedback und Kritik am Praktikum}

\paragraph{Allgemein:} Es machte uns sehr viel Spaß diese Anwendung zu entwickeln. Ein gutes Klima im Team trug dazu bei. 
Aufgrund des Unterschiedes in den Bereichen der praktischen Erfahrung konnte bei uns auch gut Wissen ausgetauscht werden, beziehungsweise entwickelte sich mit der Zeit eine Art Mentoring.
Bei der Umsetzung wurde sehr deutlich, wie das Praktikum die Vorlesungsinhalte vertieft und im direkten Bezug dazu tritt. Das Wissen wurde dadurch anschaulicher und praktisch Umgesetzt. 


\paragraph{Spezifikation der Anforderungen:} Durch die in manchen Teilen unspezifische Aufgabenstellung lässt sich schwer Abschätzen, was es genau zu erreichen gilt. Positiv daran ist die Möglichkeit der freien Auswahl der Vorgehensweise, Umsetzung und Tools. 