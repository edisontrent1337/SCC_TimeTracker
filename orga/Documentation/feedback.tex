% Feedback + Kritik am Praktikum

\section{Feedback und Kritik am Praktikum}

\paragraph{Allgemein:} Insgesamt machte es viel Spaß, diese Anwendung zu entwickeln. Ein gutes Klima im Team trug maßgeblich dazu bei. 
Aufgrund des unterschiedlicher praktischer Erfahrung fand ein reger Wissensaustausch statt. Es entwickelte sich mit der Zeit eine Art Mentoring zwischen beiden Teilnehmern.
Bei der Umsetzung wurde sehr deutlich, wie das Praktikum die Vorlesungsinhalte vertieft und in direkten praktischen Bezug dazu tritt. Das in der Vorlesung erlangte Wissen konnte anschaulich umgesetzt werden. Die im Rahmen des Praktikums entstandene Anwendung wird weiterentwickelt und genutzt werden.


\paragraph{Spezifikation der Anforderungen:} Durch die in manchen Teilen unspezifische Aufgabenstellung ließ sich schwer abschätzen, welche Anforderungen genau erfüllt werden müssen.
Positiv daran ist die Möglichkeit der freien Auswahl von Vorgehensweise, Umsetzung und Tools.
Die Zwischenpräsentation und das Feedback darauf hat jedoch maßgeblich zum besseren Verständnis der Aufgabenstellung und fokussierterem Arbeiten geführt.